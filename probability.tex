\title{Probability Study Guide}
\date{\today}

\documentclass[12pt]{article}
\usepackage{amsmath}
\begin{document}
	\maketitle
	
\section{Combinatorics}
If $r$ experiments are to be performed such that each has $n$ outcomes, there is a total of $n_1, n_2, ..., n_r$ possible outcomes of the $r$ experiments. We use the factorial to find how many different ordered arrangements are possible. When order does not matter, we use the combination formula:
\begin{align*}
  C(n,k) = \frac{n!}{(n-r)!r!}
\end{align*}

\section{Axioms}
\begin{itemize}
	\item The sample space is the set of all possible outcomes
	\item An event is a subset of the sample space:
	\begin{align*}
		P(E) = \frac{|E|}{|S|}
	\end{align*}
	\item $E\cup F$ consists of all outcomes of E and F
	\item $E\cap F=EF$ consists of all outcomes that are in both E and F
	\item if $EF = \emptyset$, E and F do not have any like outcomes and are mutually exclusive
	\item $E^c$ is the complement of E that contains everything in sample space not in E
	\item DeMorgan's Laws $(\cup E_i)^c = \cap E^c_i$
	\item $0 \le P(E) \le 1$
	\item $P(S) = 1$
	\item $P(\cap E_i) = \sum P(E_i)$
\end{itemize}

\section{Propositions}
\begin{itemize}
	\item $P(E^c) = 1 - P(E)$
	\item if $E\subset F$, then $P(E) \le P(F)$
	\item $P(E \cup F) = P(E) + P(F) - P(EF)$
	\item $P(\bigcup^n E_i) = \Sigma P(E_i) - \Sigma P(E_{i_1} E_{i_2}) + ... + (-1)^{n+1} P(E_1 E_2 ... E_n)$
\end{itemize}

\section{Bayes' Theorem}
Bayes' theorem describes the probability of an event based on prior knowledge of conditions related to the event.
\begin{align*}
	P(A|B) = \frac{P(B|A)P(A)}{P(B)}
\end{align*}
where $P(A)$ is the prior initial belief in event $A$, $P(A|B)$ is the posterior belief accounting for $B$, and $\frac{P(B|A)}{P(B)}$ is the support that B provides for A. The $P(B)$ can be expanded to
\begin{align*}
	P(B) = P(B|A)P(A) + P(B|A^c)P(A^c)
\end{align*}

\section{Discrete Random Variables}
A random variable $X$ is a function from a sample space to the real number. With discrete RV's, $X$ take s in countably infinite set of numbers. The probability mass function gives the probability a discrete RV is exactly equal to some value, represented $P(X=i)$. The cumulative distribution function,$F(x) = P(X\le x)$, is the probability a random variable is random variable is equal to or less than some value.

\subsection{Expected Value}
The expected value of a random variable is the long run average value of repetitions of the experiment it represents.

\begin{align*}
	E(x) = \sum_{i=1}^{n} x_i p_i\\
	E(g(x)) = \sum g(x_i)p(x_i)\\
	E(ax+ b) = aE(x) + b
\end{align*} 

\subsection{Variance}
Variance measures how far a set of random numbers are spread out from their mean.
\begin{align*}
	var(x) = \sum_i^2 = E(x^2) - [E(x)]^2\\
	var(x+b) = v(x)\\
	var(ax) = a^2 var(x)
\end{align*}

\subsection{Moment Generating Function}
The expected values $E(X), E(X^2), ..., E(X^n)$ are called moments. The moment generating function is defined as:
\begin{align*}
	M(t) &= E(e^{tx}) = \sum e^{tx}f(x) \\
	m^{'}(0) &= E(X) \\
	m^{''}(0) &= E(X^2)
\end{align*}

If $M_1 (t) = M_2 (t)$ then random variables $X_1 = X_2$

\subsection{Discrete Distributions}
\subsubsection{Binomial (Bernoulli) Random Variable}
The binomial distribution expresses the number of successes in a sequence of n independent binary experiments.
\begin{align*}
	x_i = \begin{cases}
	1 success \\
	0 failure
	\end{cases} \\
	P(X_i = 1) = p, P(X_i=0) = 1 - p \\
	P(X=k) = \dbinom{n}{k}p^k (1-p)^{n-k} \\
	E(x)=np, var(x) = npq, \: \text{where} \: q=(1-p)
\end{align*}

\subsubsection{Poisson Distribution}
The Poisson distribution expresses the probability of a given number of events occurring in a fixed interval of time if the events occur with a known average rate, $\lambda$, and independently. 
\begin{align*}
	P(X=k) = \frac{\lambda^k}{k!}e^{-\lambda} \\
	E(x) = var(x) = \lambda	
\end{align*}

\subsubsection{Geometric Distribution}
The geometric distribution expresses the probability of $n$ Bernoulli trials yields the first success.
\begin{align*}
	P(X=n) &= (1 - p)^{n-1} p\\
	E(x) &= \frac{1}{p} \\
	var(x) &= \frac{1-p}{p^2}
\end{align*}
The cumulative distribution function express the probability of at least $k$ trials before first success.
\begin{align*}
P(X\ge k) = (1-p)^{k-1}
\end{align*}

\subsubsection{Negative Binomial Distribution}
The negative binomial distribution expresses the probability it takes $k$ trials until $r$th success.
\begin{align*}
	P(X=k) &= \binom{k-1}{r-1}p^r (1-p)^{k-r} \\
	E(x) &= \frac{r}{p} \\
	var(x) &= \frac{r(1-p)}{p^2}
\end{align*}

\subsubsection{Hypergeometric Distribution}
The hypergeometric distribution expresses the probability of $k$ successes in $n$ draws, without replacement, from a population of size $N$ and $k$ successes.
\begin{align*}
	P(X=k) &= \frac{\binom{K}{k}\binom{N-K}{n-k}}{\binom{N}{n}} \\
	E(X) &= \frac{nm}{N} \\
	var(X) &= npq\binom{N-m}{N-1}
\end{align*} 

\section{Continuous Random Variable}
$X$ is a continuous random variable if there exists a function $f(x)$ such that $f(x)$ is the non-negative domain of $x\in (-\infty, \infty)$ such that $P(X\in B) = \int_B f(x)dx$.
\begin{align*}
	P(-\infty\le x \le\infty) = 1 \\
	P(X = a) = \int_{a}^{a}f(x)dx=0 \\
	P(X<a) = \int_{-\infty}^{a}f(x)dx = F(a) \\
	E(X) = \int_{-\infty}^{\infty}xf(x)dx \\
	var(X) = \int_{-\infty}^{\infty}x^2 f(x)dx - \bigg(\int_{-\infty}^{\infty}x f(x)dx\bigg)^2 \\
	M(t) = E(e^{tx}) = \int_{-\infty}^{\infty}e^{tx} f(x)dx
\end{align*}

\subsection{Continuous Distribution}
\subsubsection{Uniform Distribution}
The uniform distribution is uniformly distributed on interval $[a, b]$ if it assumes all values on $[a, b]$. It is represented by the probability density function:
\begin{align*}
	f(x) = \begin{cases}
		\frac{1}{b-a}, a\le x\le b \\
		0, \text{otherwise}
	\end{cases}
\end{align*}
 and the cumulative distribution function:
 \begin{align*}
	 F(x)=P(X\le x)= \begin{cases}
		 0, x\le a \\
		 \frac{x-a}{b-a}, a\le x\le b \\
		 1, otherwise
	 \end{cases}
 \end{align*}
 
 \subsubsection{Exponential Distribution}
 


\end{document}