\title{Linear Algebra Study Guide}
\date{\today}

\documentclass[12pt]{article}
\usepackage{amsmath}
\usepackage{amssymb}
\begin{document}
	\maketitle
	
\section{Matrices}
$A$ is an $m \times n$, with $m$rows and $n$columns, $A = (A(j,k))_{j=1:m, k=1:n}$ or in the column form $A=col(A_k)_{k=1:n}$.

\section{Matrix Vector Multiplication}
$A=col(a_k)_{k=1:n}$, $v = (v_i)_{i=1:n}$ is a column vector of size $n$ then $Av=a_1v_1+a_2v_2+...+a_nv_n=\sum_{i=1}^{n}v_ka_k$. $Av$ is a linear combination of the columns of $A$ with coefficients the entries of $v$.

\section{Matrix Matrix Multiplication}
$A$ is $m\times n$, $B$ is $n\times p$ with $B=col(b_k)_{k=1:p}$ then $AB$ is $m\times p$ and $AB=col(Ab_k)_{k=1:p}.$

\section{Row-Vector Column-Vector Multiplication}
$w^t = (w_1,w_2,...,w_n)$, $v=\begin{bmatrix}
	v_1 \\ v_2 \\ ... \\ v_n
\end{bmatrix}$ then $w^tv = \sum_{i=1}^{n}w_iv_i$

\section{Matrix Matrix Matrix Multiplication}
$A$ is $m\times n$ with $A=row(r_j)_{j=1:m}$, $B$ is $n \times p$, $C$ is $p\times l$ with $C=col(c_k)_{k=1:p}$ then $ABC$ is $m\times l$ with $(ABC)(j,k) = r_jBc_k$

\section{Transpose of a Matrix}
$A$ is $m\times n$, then $A^t$ is $n\times m$ and $A^t(j,k)=A(k,j)$
\begin{itemize}
	\item $(AB)^t = B^tA^t$
	\item $(Av)^t = v^tA^t$
\end{itemize}

\section{Identity Matrix}
$I = \begin{bmatrix}
	1 & 0 & ... & 0 \\
	0 & 1 & ... & 0 \\
	| & ...  & ... & | \\
	0 & ... & ... & 1
\end{bmatrix}$ column form $I=col(c_k)_{k=1:n}$ where $c_k = \begin{Bmatrix}
	0 \\ 0 \\ ... \\ 0 \\ 1 \\ 0 \\ ... \\ 0
\end{Bmatrix}$ with the $1$ being in the $k$th position.  \\
$AI=IA=A$ \\
$A$ is symmetric iff $A^t=A$

\section{Matrix Rank, Nullspace, Range of a Matrix}
\begin{itemize}
	\item Definition: Vectors $w_1,w_2,...,w_n$ (of the same size) are linearly independent iff $\sum c_iw_i = 0 <=> c_i =0, \forall_i = 1:p$
	\item Definition: Let $A$ be an $m\times n$ matrix then the rank of $A$ is equal to the largest number of linearly independent columns of $A$ (which is equal to row rank).
	\item Definition: The nullspace of $A$ is $Null(A)={v\in \mathbb{R}^n:Av=0}$
	\item Definition: The range of A is $Range(A)={u\in\mathbb{R}^m:\, there\, exsists\, v\in\mathbb{R}^n\, with\, Av=u}$
	\item Recall: If $A = col(a_k)_{k=1:n}$ and $v=(v_i)_{i=1:n}$ then $Av=\sum_{k=1}^{n}v_ka_k$
	\item Definition: The space generated by vectors $w_1,w_2,...,w_n$  then  $<w_1,w_2,...,w_n>=\{\sum_{i=1}^{p}\, with\, c_i\in\mathbb{R}\}$ . The range of a matrix is equal to the space generated by the columns of the matrix.
	\item Definition: The dimension of a vector space in $\mathbb{R}^n$ is equal to the smallest number of linearly independent vectors from this space such that the space generated by their linearly independent vectors is equal to the vector space. These linearly independent vectors form a basis for the vector space.
\end{itemize}

\section{Non-Singular Matrices}
\begin{itemize}
	\item Definition: Let $A$ be a square matrix of size $n$. $A$ is nonsingular iff there exists a square matrix $A_{-1}$ of size $n$ such that $AA_{-1}=A^{-1}A=I$
	\item Theorem: $A$ is nonsingular if $Null(A)=\{0\}, Range(A)=\mathbb{R}^n, Rank(A)=n, det(A)\ne 0$
\end{itemize}

\section{Determinants}
The determinant can be viewed as the scaling factor of the transformation described by the matrix.
\begin{align*}
	det\begin{bmatrix}
		a & b \\
		c & d 
	\end{bmatrix}&=ad-bc \\
	det(AB) &= det(A)det(B) \\ 
	det\begin{bmatrix}
		d_1 &  & 0 \\
		 & \ddots & \\
		0 & & d_k
	\end{bmatrix} &= \prod_{i=1}^{n}d_i
\end{align*}\\
Lemma: If $AB=I$, theN $A$ is nonsingular, $BA=I$ then $B=A^{-1}$.

\section{Diagonal Matrices}




\end{document}